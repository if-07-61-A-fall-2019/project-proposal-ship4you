\documentclass[12pt]{article}
\usepackage{geometry}                % See geometry.pdf to learn the layout options. There are lots.
\geometry{letterpaper}                   % ... or a4paper or a5paper or ... 
\usepackage{graphicx}
\usepackage{amssymb}
\usepackage{amsthm}
\usepackage{epstopdf}
\usepackage[utf8]{inputenc}
\usepackage[usenames,dvipsnames]{color}
\usepackage[table]{xcolor}
\usepackage{hyperref}
\DeclareGraphicsRule{.tif}{png}{.png}{`convert #1 `dirname #1`/`basename #1 .tif`.png}

\theoremstyle{definition}
\newtheorem{example}{Example}

\newenvironment{explanation}{%
   \setlength{\parindent}{0pt}
   \itshape
   \color{blue}
}{}

\newcommand{\projectname}{Ship4You}
\newcommand{\productname}{Ship4You}
\newcommand{\projectleader}{Alexander Hrazdera/Paul Schoeffl}
\newcommand{\documentstatus}{In Arbeit}
%\newcommand{\documentstatus}{Submitted}
%\newcommand{\documentstatus}{Released}
\newcommand{\version}{V. 1.0}

\begin{document}
\begin{titlepage}

\vspace{10em}

\begin{center}
{\Huge Project Proposal} \\[3em]
{\LARGE \productname} \\[3em]
\end{center}

\begin{flushleft} 
\begin{tabular}{|l|l|}
\hline
Projekt Name & \projectname \\ \hline
Projekt Leiter & \projectleader \\ \hline
Dokumenten Status & \documentstatus \\ \hline
Version & \version \\ \hline
\end{tabular}
\end{flushleft}

\end{titlepage}
\section*{Revisions}
\begin{tabular}{|l|l|l|}
\hline
\cellcolor[gray]{0.5}\textcolor{white}{Date} & \cellcolor[gray]{0.5}\textcolor{white}{Author} & \cellcolor[gray]{0.5}\textcolor{white}{Change} \\ \hline
September 19, 2019&A. Hrazdera/P. Schoeffl&First version \\ \hline
\end{tabular}
\pagebreak

\section*{Contents}
\begin{itemize}
\item Einfuehrung
\item Ausgangssituation
\item Allgemeine Bedingungen und Einschraenkungen
\item Projektziele und Systemkonzepte
\item Chancen und Risiken
\item Planung
\end{itemize}
\pagebreak
\end{article}
\section{Einfuehrung}
\begin{article}
Das Ziel dieses Projektes ist es, eine Website fuer Segel oder Motoryachten (Boote) zu erstellen. Der Sinn dieser Website ist es, die Boote zur Bewertung freizugeben. Somit kann jeder User der sich ein Boot mieten moechte (egal wo) ein gutes Bild ueber den Vermieter und den Zustand des Bootes machen. Somit kann es nicht mehr passieren, dass man von einem Vermieter (Charterer) ueber den Tisch gezogen wird. Auch kann man sich so leichter mit den Vermietern zusammenschreiben. Weil es so ein Produkt noch nicht auf dem Markt gibt kann man die Idee auch leicht erweitern und vermarkten. 

\end{article}
\pagebreak

\section{Ausgangssituation}
\begin{article}
   Da es so ein Produkt noch nicht auf dem Markt gibt, haelt sich die Konkurrenz in Grenzen. Auch kann man so ein Prdoukt sehr gut gebrauchen. Zweitens haben wir schon von Leuten gehört, die schon nach einer solchen Website gesucht haben. Diese haben uns auch gebeten so eine zu Programmieren. 
\end{article}

\pagebreak

\section{Allgemeine Bedingungen und Einschraenkungen}

\begin{article}
Das vorgeschlagene System hat mit den folgenden Einschraenkungen zu kaempfen:
\begin{itemize}
\item Die Benutzerdaten sind streng vertraulich $\rightarrow$ Da wir persoenliche Daten wie (Name, Email, Wohnort und Geburtsjahr) bei angemeldeten Kunden speichern.
\item Die Benutzeroberflaeche des Informationssystems muss leicht verständlich und modern gestaltet sein. Man sollte auf dem ersten Blick sofort wissen was man genau machen muss und wo man nach den verschiedenen Schiffen suchen muss (Ohne Anmeldung).
\item Die Anwendung muss mehrsprachig gestaltet werden, weil viele Haefen nicht in Oesterreich und Deutschland sind. Am Anfang wird die Website 2 - Sprachig ausgelegt sein (Deutsch und Englisch).
\end{itemize}
\end{article}

\pagebreak

\section{Projektziele und Systemkonzepte}
\begin{article}
Die Projektziele lassen sich wie folgt zusammenfassen:
\begin{itemize}
\item Man muss nicht angemeldet sein um sich die Bewertungen zu den Booten anzusehen. Somit ist die Bedienung fuer den Kunden leichter bzw. angenehmer, weil viele Kunden sich nicht nur fuers ansehen anmelden moechten.
\item Anmeldung nur wenn man neue Boote hinzufuegen, oder Kommentare verfassen moechte. Hier ist eine Anmeldung wichtig damit man sieht wer diese Nachricht gesendet hat und somit auch Bots vermieden werden koennen.
Auch zum Bewerten der Boote muss man sich mit Email, Passwort, Vorname, Nachname und Geburtsdatum registrieren.
Hierbei ist eine Anmeldung jedoch wichtig weil man somit Bots vermeiden kann.
\item Schnelles Fragenstellen moeglich, somit koennen sich auch die User untereinander austeilen.
\item Bei unserer Website steht der Kunde im Vordergrund und das soll auch durch einen kompetenten und freundlichen Support ausgedrueckt werden. 
\end{itemize}
\end{article}

\pagebreak
\section{Chancen und Risiken}
\begin{article}
Das Projekt hat folgende Moeglichkeiten:
\begin{itemize}
\item Mieter bekommt fuer sein Geld das bestmoegliche Boot, weil er von anderen Usern die Bewertungen der Boote einsehen kann.
\item Risiko betrogen zu werden sinkt, weil andere User diese Boote schon getestet haben und ihre Bewertungen zum Charterer(Vermieter) abgegeben haben.
\item Man kann sich einen Ueberblick ueber das Boot machen. Viele Kommentare und Bilder werden vom aktuellen Zustand des Bootes zu sehen sein.
\end{itemize}

Das folgende Risiko ist zu beruecksichtigen:
\begin{itemize}
\item Das nicht viele Boote hinzugefuegt werden. Am Anfang ist es sicherlich schwer diese Website bekannter zu machen. Die Vermarktung ist darum eine potentielle Schwäche.
\item Aufpassen muss man auch das die Bewertungen der Menschen nicht durch Bots verfaelscht werden. (Darum auch die Anmeldung)
\end{itemize}

\end{article}

\pagebreak
\section{Planung}
\begin{article}
Der erste Meilenstein wird die Erstellung der Haupseite sein. Auf dieser Seite kann der User nach dem Standort und Namen seines Bootes suchen. Es werden ihm sein Boot, die Bewertungen bzw. Kommentare dazu auf dem ersten Blick angezeigt. Nun kann er sich mit anderen Leuten darüber austauschen. Auch kann er sich bei Fragen an die Community oder direkt an den Charterer wenden. 
Der zweite Meilenstein wird das Anmelden der Benutzer sein. Dieser Punkt ist fuer die Vermieter wichtig. Er kann somit sein Boot auf diese Website stellen. Auch ist diese Funktion wichtig fuer Benutzer die das Boot schon gemietet haben und eine Bewertung fuer dieses Boot abgeben moechten.
Als drittes folgt eine Datenbank mit den ganzen Haefen bzw. Booten, hierbei ist es wichtig, das die Daten schnell und ohne Probleme zur Verfuegung stehen. Der vierte Meilenstein ist fuer die Suche dieser Haefen bzw. Boote zustaendig. Auch hier darf die Suche nicht zu lange dauern. Um diese Daten zu finden wird die Datenbank durchsucht.\\ Alexander Hrazdera und Paul Schoeffl sind fuer die Dokumentation und der Erstellung der Website zustaendig.\\ Wir benoetigen 2 Programmierer mit jeweils einem eigenen Notebook. Einen Server mit einer Datenbank und eine Websitedomain wird auch benoetigt.
\end{article} 

\newcommand{\projektend}{30 Juni 2020}
\newcommand{\projectstart}{23 September 2019}
\newcommand{\firstresult}{Ende Jaenner}
\newcommand{\beginofprog}{7 Oktober}
\newcommand{\bigBlocks}{Design der Website/der Login/Datenbankerstellung}
\newcommand{\whatisneeded}{Datenbank/Websitedomaein}

\begin{flushleft} 
\begin{tabular}{|l|l|}
\hline
Projekt Ende: & \projektend \\ \hline
Projekt Start: & \projectstart \\ \hline
Erster fertiger Prototyp: & \firstresult \\ \hline
Beginn der Umsetzung: & \beginofprog \\ \hline
Groessten Bloecke: & \bigBlocks \\ \hline
Was wird benoetigt? & \whatisneeded \\ \hline
\end{tabular}
\end{flushleft}

\end{document}  