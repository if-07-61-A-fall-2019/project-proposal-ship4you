\documentclass[12pt]{article}
\usepackage{geometry}                % See geometry.pdf to learn the layout options. There are lots.
\geometry{letterpaper}                   % ... or a4paper or a5paper or ... 
\usepackage{graphicx}
\usepackage{amssymb}
\usepackage{amsthm}
\usepackage{epstopdf}
\usepackage[utf8]{inputenc}
\usepackage[usenames,dvipsnames]{color}
\usepackage[table]{xcolor}
\usepackage{hyperref}
\DeclareGraphicsRule{.tif}{png}{.png}{`convert #1 `dirname #1`/`basename #1 .tif`.png}

\theoremstyle{definition}
\newtheorem{example}{Example}

\newenvironment{explanation}{%
   \setlength{\parindent}{0pt}
   \itshape
   \color{blue}
}{}

\newcommand{\projectname}{Ship4You}
\newcommand{\productname}{Ship4You}
\newcommand{\projectleader}{Alexander Hrazdera}
\newcommand{\documentstatus}{In Arbeit}
%\newcommand{\documentstatus}{Submitted}
%\newcommand{\documentstatus}{Released}
\newcommand{\version}{V. 1.0}

\begin{document}
\begin{titlepage}
\begin{flushright}
\includegraphics[scale=.5]{htlleondinglogo.png}\\
\end{flushright}

\vspace{10em}

\begin{center}
{\Huge Project Proposal} \\[3em]
{\LARGE \productname} \\[3em]
\end{center}

\begin{flushleft}
\begin{tabular}{|l|l|}
\hline
Project Name & \projectname \\ \hline
Project Leader & \projectleader \\ \hline
Document state & \documentstatus \\ \hline
Version & \version \\ \hline
\end{tabular}
\end{flushleft}

\end{titlepage}
\section*{Revisions}
\begin{tabular}{|l|l|l|}
\hline
\cellcolor[gray]{0.5}\textcolor{white}{Date} & \cellcolor[gray]{0.5}\textcolor{white}{Author} & \cellcolor[gray]{0.5}\textcolor{white}{Change} \\ \hline
September 19, 2019&A. Hrazdera/P. Sch�ffl&First version \\ \hline
\end{tabular}
\pagebreak

\tableofcontents
\pagebreak

\section{Einfuehrung}
\begin{explanation}
Das Ziel dieses Projektes ist es, eine Website fuer Segel oder Motoryachten (Boote) zu erstellen. Der Sinn dieser Website ist es, die Boote zur Bewertung freizugeben. Somit kann jeder User der sich ein Boot mieten m�chte (egal wo) ein gutes Bild �ber den Vermieter und den Zustand des Bootes machen. Somit kann es nicht mehr passieren das man von einem Vermieter (Charterer) �ber den Tisch gezogen wird. Auch kann man sich so leichter mit den Vermietern zusammenschreiben. Weil es so ein Produkt noch nicht auf dem Arbeitsmarkt gibt kann man die Idee auch leicht erweitern und vermarkten. 

\end{explanation}
\pagebreak

\section{Ausgangssituation}
\begin{explanation}
Da wir uns schon etwas mit html, css und JavaScript auskennen wird der Anfang fuer uns nicht so schwer werden.
Jedoch koennen wir Angular noch nicht und muessen uns damit erst beschaeftigen. Darum werden wir uns werend dem Programmieren und davor mit Angular auseinander setzen. Auch kennen wir uns im Team schon, weil wir schon ein anderes Projekt gemeinsam hatten. Somit koennen wir die Kennenlernphase auslassen.
\end{explanation}

\pagebreak

\section{Allgemeine Bedingungen und Einschraenkungen}

\begin{explanation}
Das vorgeschlagene System hat mit den folgenden Einschraenkungen zu kaempfen:
\begin{itemize}
\item Die Benutzerdaten sind streng vertraulich.
\item Die Benutzeroberflaeche des Informationssystems muss intuitiv sein.
\item Die Anwendung ist mehrsprachig (englisch und deutsch).
\end{itemize}
\end{explanation}

\pagebreak

\section{Projektziele und Systemkonzepte}
\begin{explanation}
Die Projektziele lassen sich wie folgt zusammenfassen:
\begin{itemize}
\item Man muss nicht angemeldet sein um sich die Bewertungen zu den Booten anzusehen.
\item Anmeldung nur wenn man neue Boote hinzufuegen, oder Kommentare verfassen moechte.
Auch zum Bewerten der Boote muss man sich mit Email, Passwort, Vorname, Nachname und Geburtsdatum registrieren.
\item Schnelles Fragenstellen moeglich.
\end{itemize}
\end{explanation}

\pagebreak
\section{Chancen und Risiken}
\begin{explanation}
Das Projekt hat folgende Moeglichkeiten:
\begin{itemize}
\item Mieter bekommt fuer sein Geld das bestmoegliche Boot.
\item Risiko betrogen zu werden sinkt.
\item Man kann sich einen Ueberblick ueber das Boot machen.
\end{itemize}

Das folgende Risiko ist zu beruecksichtigen.
\begin{itemize}
\item Das nicht viele Boote hinzugefuegt werden.
\item Das die Bewertungen verfaelscht werden.
\end{itemize}

\end{explanation}

\pagebreak
\section{Planung}
\begin{explanation}
Der erste Meilenstein wird die Erstellung der Haupseite sein.
Der zweite Meilenstein wird das Anmelden der Benutzer sein.
Als drittes folgt eine Datenbank mit den ganzen Haefen bzw. Booten.
Der vierte Meilenstein ist fuer die Suche dieser Haefen bzw. Boote zustaendig.

Alexander Hrazdera ist fuer die Dokumentation und der Erstellung der Website zustaendig.
Paul Schoeffl ist auch fuer die Dokumentation und der Erstellung der Website zustaendig. 

Wir benoetigen 2 Programmierer mit jeweils einem eigenen Notebook. Einen Server mit einer Datenbank und eine Websitedomain wird auch benoetigt.

Das Projekt wird am 30 Juni 2020 enden.
Das Projekt wird am 23 September 2019 starten. 
\begin{itemize}
\item Das Projekt wird am 30 Juni 2020 enden.
\item Das Projekt wird am 23 September 2019 starten.
\item Der erste Prototyp wird ende Jaenner zur Verfuegung stehen.
\item Die Umsetzung wird in den naechsten Wochen beginnen.
\item Das Design der Website, der Login, die Datenbankerstellung und die Suche der Boote sind die grossen Bloecke unserer Arbeit.
\item Die Arbeit wird bis zum 30 Juni beendet sein.
\item Fuer unsere Arbeit benoetigen wir eine Datenbank und eine Website Domain.
\end{itemize}
\end{explanation}

\end{document}  
