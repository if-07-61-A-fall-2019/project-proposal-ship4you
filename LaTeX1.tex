\documentclass[12pt]{article}
\usepackage{geometry}                % See geometry.pdf to learn the layout options. There are lots.
\geometry{letterpaper}                   % ... or a4paper or a5paper or ... 
\usepackage{graphicx}
\usepackage{amssymb}
\usepackage{amsthm}
\usepackage{epstopdf}
\usepackage[utf8]{inputenc}
\usepackage[usenames,dvipsnames]{color}
\usepackage[table]{xcolor}
\usepackage{hyperref}
\DeclareGraphicsRule{.tif}{png}{.png}{`convert #1 `dirname #1`/`basename #1 .tif`.png}

\theoremstyle{definition}
\newtheorem{example}{Example}

\newenvironment{explanation}{%
   \setlength{\parindent}{0pt}
   \itshape
   \color{blue}
}{}

\newcommand{\projectname}{Ship4You}
\newcommand{\productname}{Ship4You}
\newcommand{\projectleader}{Alexander Hrazdera}
\newcommand{\documentstatus}{In Arbeit}
%\newcommand{\documentstatus}{Submitted}
%\newcommand{\documentstatus}{Released}
\newcommand{\version}{V. 1.0}

\begin{document}
\begin{titlepage}
\begin{flushright}
\includegraphics[scale=.5]{htlleondinglogo.png}\\
\end{flushright}

\vspace{10em}

\begin{center}
{\Huge Project Proposal} \\[3em]
{\LARGE \productname} \\[3em]
\end{center}

\begin{flushleft}
\begin{tabular}{|l|l|}
\hline
Project Name & \projectname \\ \hline
Project Leader & \projectleader \\ \hline
Document state & \documentstatus \\ \hline
Version & \version \\ \hline
\end{tabular}
\end{flushleft}

\end{titlepage}
\section*{Revisions}
\begin{tabular}{|l|l|l|}
\hline
\cellcolor[gray]{0.5}\textcolor{white}{Date} & \cellcolor[gray]{0.5}\textcolor{white}{Author} & \cellcolor[gray]{0.5}\textcolor{white}{Change} \\ \hline
September 19, 2019&A. Hrazdera/P. Sch�ffl&First version \\ \hline
\end{tabular}
\pagebreak

\tableofcontents
\pagebreak

\section{Einf�hrung}
\begin{explanation}
Das Ziel dieses Projektes ist es, eine Website f�r Segel oder Motoryachten (Boote) zu erstellen. Der Sinn dieser Website ist es, die Boote zur Bewertung freizugeben. Somit kann jeder User der sich ein Boot mieten m�chte (egal wo) ein gutes Bild �ber den Vermieter und den Zustand des Bootes machen. Somit kann es nicht mehr passieren das man von einem Vermieter (Charterer) �ber den Tisch gezogen wird. Auch kann man sich so leichter mit den Vermietern zusammenschreiben. Weil es so ein Produkt noch nicht auf dem Arbeitsmarkt gibt kann man die Idee auch leicht erweitern und vermarkten. 

\end{explanation}
\pagebreak

\section{Ausgangssituation}
\begin{explanation}
Da wir uns schon etwas mit html, css und JavaScript auskennen wird der Anfang f�r uns nicht so schwer werden.
Jedoch k�nnen wir Angular noch nicht und m�ssen uns damit erst besch�ftigen. Darum werden wir uns werend dem Programmieren und davor mit Angular auseinander setzen. Auch kennen wir uns im Team schon, weil wir schon ein anderes Projekt gemeinsam hatten. Somit k�nnen wir die kennenlern Phase auslassen.
\end{explanation}

\pagebreak

\section{General Conditions and Constraints}

\begin{example}
Das vorgeschlagene System hat mit den folgenden Einschr�nkungen zu k�mpfen:
\begin{itemize}
\item Die Benutzerdaten sind streng vertraulich.
\item Die Benutzeroberfl�che des Informationssystems muss intuitiv sein.
\item Die Anwendung ist mehrsprachig (englisch und deutsch).
\end{itemize}
\end{example}

\pagebreak

\section{Projektziele und Systemkonzepte}
\begin{example}
The project objectives can be summarized as follows:
\begin{itemize}
\item Man muss nicht angemeldet sein um sich die Bewertungen zu den Booten anzusehen.
\item Anmeldung nur wenn man neue Boote hinzuf�gen, oder Kommentare verfassen m�chte.
Auch zum Bewerten der Boote muss man sich mit Email, Passwort, Vorname, Nachname und Geburtsdatum registrieren.
\item Schnelles Fragenstellen m�glich
\end{itemize}
\end{example}

\pagebreak
\section{Chancen und Risiken}
\begin{example}
Das Projekt hat folgende M�glichkeiten:
\begin{itemize}
\item Mieter bekommt f�r sein Geld das bestm�gliche Boot
\item Risiko betrogen zu werden sinkt.
\item Man kann sich einen �berblick �ber das Boot machen.
\end{itemize}

Das folgende Risiko ist zu ber�cksichtigen.
\begin{itemize}
\item Das nicht viele Boote hinzugef�gt werden
\item Das die Bewertungen verf�lscht werden
\end{itemize}

\end{example}

\pagebreak
\section{Planung}
\begin{explanation}
Der erste Meilenstein wird die erstellung der Haupseite sein.
Der zweite Meilenstein wird das Anmelden der Benutzer sein.
Als drittes eine Datenbank mit den ganzen H�fen bzw. Booten.
Der vierte Meilenstein ist f�r die Suche dieser H�fen bzw. Boote zust�ndig.

Alexander Hrazdera ist f�r die Dokumentation und der Erstellung der Website zust�ndig.
Paul Sch�ffl ist auch f�r die Dokumentation und der Erstellung der Website zust�ndig. 

Wir ben�tigen 2 Programmierer mit jeweils einem eigenen Notebook. Einen Server mit einer Datenbank und eine Websitedomain wird auch ben�tigt.

Das Projekt wird am 30 Juni 2020 enden.
Das Projekt wird am 23 September 2019 starten. 
Answer the following questions when preparing this section:
\begin{itemize}
\item Das Projekt wird am 30 Juni 2020 enden.
\item Das Projekt wird am 23 September 2019 starten.
\item Der erste Prototyp wird ende J�nner zur Verf�gung stehen.
\item Die Umsetzung wird in den n�chsten Wochen beginnen.
\item Das Design der Website, der Login, die Datenbank erstellung und die Suche der Boote sind die gro�en Bl�cke unserer Arbeit.
\item Die Arbeit wird bis zum 30 Juni beendet sein.
\item F�r unsere Arbeit ben�tigen wir eine Datenbank und eine Website Domain.
\end{itemize}
\end{explanation}

\end{document}  